\hyperlink{interface_j_p_color}{JPColor} is an Color Object that encapsulate information about colors on different color spaces. 

Is similar to the \href{http://developer.apple.com/library/ios/#documentation/uikit/reference/UIColor_Class/Reference/Reference.html}{\tt UIKit/UIColor class}, but with a lot of more convenient and powerfull features. Contains several convenient methods to easily create colors and convert between spaces.

\subsection*{Creating colors }

An example to how to create an \hyperlink{interface_j_p_color}{JPColor} with RGB values. 
\begin{DoxyCode}
 JPColor anColor = [JPColor initWithRed:255 G:0 B:0 opacity:100];
\end{DoxyCode}
 The above code create an {\bfseries red} color. Note that \hyperlink{interface_j_p_color}{JPColor} use an Photoshop-\/like values (0-\/255) to represent RGB colors. So once created an \hyperlink{interface_j_p_color}{JPColor} is easy to convert it to an UIColor like this: 
\begin{DoxyCode}
 UIColor anUIColor = [[JPColor initWithRed:255 G:0 B:0 opacity:100] UIColor];
\end{DoxyCode}
 You also can create color using different color spaces, like CMYK; 
\begin{DoxyCode}
 JPColor *cmyk = [JPColor initWithCyan:0 M:100 Y:100: K:0 opacity:100];
\end{DoxyCode}
 Or an CSS string, like this: 
\begin{DoxyCode}
 JPColor *color = [JPColor initWithCSS:@"#2658e8"];
\end{DoxyCode}


\subsection*{Colors Templates}

\hyperlink{interface_j_p_color}{JPColor} includes an collection of color templates methods that follow the \href{http://www.w3.org/TR/css3-color/#svg-color}{\tt W3C CSS 3 Extended Colors Keywords}. You can retrieve this methods simply calling his name, like this: 
\begin{DoxyCode}
 JPColor *seaGreenColor = [JPColor seagreen];
\end{DoxyCode}


\subsection*{Colors Type Structures}

\hyperlink{interface_j_p_color}{JPColor} return different \hyperlink{_j_p_color_types_8h}{structures types} with values for different color spaces. See \hyperlink{interface_j_p_color_a659b3f9caaf643fc6e9643b8470014f6}{RGB}, \hyperlink{interface_j_p_color_a06d575f03ff1692f65871d0485b62ae6}{CMYK} and \hyperlink{interface_j_p_color_a00bf425621fe08bd7eb6154be6b5a8a9}{HSV} properties to learn more about it. Here one exampe retrieving an CMYK structure and displaying it: 
\begin{DoxyCode}
 JPColor *anColor = [JPColor greenColor];
 JPcmyk repr = [anColor CMYK];
 
 // Print values.
 NSLog( @"C:%i, M:%i, Y:%i, K:%i", repr.C, repr.M, repr.Y, repr.K );
\end{DoxyCode}
  